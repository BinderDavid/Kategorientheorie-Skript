\documentclass[10pt,twoside,openright]{memoir}
\usepackage[paperwidth=210mm, paperheight=297mm,bindingoffset=.75in]{geometry}
\usepackage[T1]{fontenc}
\usepackage[utf8]{inputenc}
\usepackage[ngerman]{babel}
\usepackage{tgtermes}
\usepackage[protrusion=true,expansion=true]{microtype}
\usepackage{enumitem}

\usepackage{amsthm}
\theoremstyle{definition}
\newtheorem{definition}{Definition}
\newtheorem{kommentar}{Kommentar}
\newtheorem{beispiel}{Beispiel}
\newtheorem{theorem}{Theorem}

\makeatletter
\def\maketitle{%
  \null
  \thispagestyle{empty}%
  \vfill
  \begin{center}\leavevmode
    \normalfont
    {\LARGE\raggedleft \@author\par}%
    \hrulefill\par
    {\huge\raggedright \@title\par}%
    \vskip 1cm
  \end{center}%
  \vfill
  \null
  \cleardoublepage
  }
\makeatother
\author{David Binder\\ Ingo Skupin}
\title{Definitionen Kategorientheorie}
\date{}

\newcommand{\C}{\ensuremath{\mathcal{C}}}
\newcommand{\Obj}[1]{\ensuremath{\mathtext{Obj}(#1)}}
\newcommand{\Hom}[3]{\ensuremath{\mathtext{Hom}_{#1}(#2, #3)}}
\newcommand{\id}[1]{\ensuremath{\mathtext{id}_{#1}}}
\newcommand{\Set}{\ensuremath{\mathbf{Set}}}
\newcommand{\Hask}{\ensuremath{\mathbf{Hask}}}
\newcommand{\Op}[1]{\ensuremath{#1^{\mathbf{op}}}}

\setenumerate{itemsep=1.5pt}
\begin{document}

\let\cleardoublepage\clearpage

\maketitle

\frontmatter

\tableofcontents

\mainmatter

\chapter{Kategorien}

\section{Kategorien und Beispiele}

\begin{definition}[Kategorie]
  Eine Kategorie $\C$ besteht aus:
  \begin{enumerate}
  \item Den Objekten der Kategorie $\Obj{\C}$.
  \item Für je zwei Objekte $A, B \in \Obj{C}$ aus der Kategorie, der Menge der Morphismen von $A$ nach $B$, geschrieben als $\Hom{\C}{A}{B}$.
  \item Für jedes Objekt $A \in \Obj{C}$ den Identitätsmorphismus $\id{A} \in \Hom{C}{A}{A}$.
  \item Für zwei \enquote{zusammenpassende} Morphismen $f \in \Hom{\C}{B}{C}$ und $g \in \Hom{\C}{A}{B}$ existiert die Hintereinanderausführung $f \circ g \in \Hom{\C}{A}{C}$.
  \end{enumerate}
  Zusätzlich müssen folgende drei Eigenschaften gelten:
  \begin{enumerate}
  \item Die Hintereinanderausführung von Morphismen ist assoziativ: Für $f \in \Hom{\C}{C}{D}, g \in \Hom{\C}{B}{C}, h \in \Hom{\C}{A}{B}$ gilt $(f \circ g) \circ h = f \circ (g \circ h)$.
  \item Der Identitätsmorphismus ist ein rechtes Neutralelement bezüglich Hintereinanderausführung: Für alle $f \in \Hom{\C}{A}{B}$ gilt $f \circ \id{A} = f$.
  \item Der Identitätsmorphismus ist ein linkes Neutralelement bezüglich Hintereinanderausführung: Für alle $f \in \Hom{\C}{B}{A}$ gilt $\id{A} \circ f = f$.
  \end{enumerate}
\end{definition}

\begin{beispiel}[\Set]
  Die Kategorie \Set\ besteht aus allen Mengen als Objekten, und hat alle normalen (mengentheoretischen) Funktionen als Morphismen.
\end{beispiel}

\begin{beispiel}[\Hask]
  Die Kategorie \Hask\ besteht aus den Haskell Typen als Objekten, und Haskell Funktionen als Morphismen.
\end{beispiel}

\begin{definition}[Monoid]
  Ein Monoid, geschrieben als Tripel $(A,\cdot, e)$, besteht aus
  \begin{enumerate}
  \item Der Trägermenge $A$
  \item Einer binären Operation $\cdot : A \to A \to A$.
  \item Einem Neutralelement $e$, i.e. $a \cdot e = a = e \cdot a$ für alle $a \in A$.
  \end{enumerate}
\end{definition}
\begin{beispiel}[Monoid als Kategorie]
  Eine Kategorie mit exakt einem Objekt enthält exakt die gleiche Information wie ein Monoid.
\end{beispiel}  

\begin{definition}[Partielle Ordnung]
  Eine Relation $\leq$ auf einer Menge $A$ ist eine partielle Ordnung wenn für alle $a,b,c \in A$ gilt:
  \begin{enumerate}
  \item $\leq$ ist reflexiv, i.e. $a \leq a$.
  \item $\leq$ ist transitiv, i.e. $a \leq b$ und $b \leq c$ impliziert $a \leq c$.
  \item $\leq$ ist antisymmetrisch, i.e. $a \leq b$ und $b \leq a$ impliziert $a = b$.
  \end{enumerate}
\end{definition}

\begin{beispiel}[Partielle Ordnung als Kategorie]
  Eine Kategorie $\C$ in der für je zwei Objekte $A, B \in \Obj{\C}$ die Menge \Hom{\C}{A}{B} entweder 0 oder 1 Element enthält, enthält exakt die gleiche Information wie eine partielle Ordnung.
\end{beispiel}

\begin{definition}[Duale Kategorie]
  Die duale Kategorie \Op{\C}\ zu einer Kategorie \C\ erhält man, indem man die Richtung aller Morphismen umdreht.
\end{definition}

\section{Isomorphismen, initiale und terminale Objekte}

\begin{definition}[Initiales Objekt]
  Ein Objekt $A$ in einer Kategorie \C\ ist ein initiales Objekt, falls für jedes andere Objekt $B$ exakt ein Morphismus in \Hom{\C}{A}{B}\ existiert.
\end{definition}

\begin{definition}[Terminales Objekt]
  Ein Objekt $A$ in einer Kategorie \C\ ist ein terminales Objekt, falls für jedes andere Objekt $B$ exakt ein Morphismus in \Hom{\C}{B}{A}\ existiert.
\end{definition}

\begin{beispiel}[Initiales und terminales Objekt in \Set]
  Die leere Menge ist das (einzige) initiale Objekt in \Set.
  Jede ein-elementige Menge ist ein terminales Objekt in \Set.
\end{beispiel}

\begin{beispiel}[Initiales und terminales Objekt in \Hask]
  \texttt{Void} ist das initiale Objekt in \Hask. Unit, geschrieben $()$, ist das terminale Objekt in \Hask.
\end{beispiel}

\begin{beispiel}[Dualität von initialem und terminalem Objekt]
  Jedes initiale Objekt in einer Kategorie \C\ ist ein terminales Objekt in der dualen Kategorie \Op{\C}, und vice versa.
\end{beispiel}
  
\begin{definition}[Isomorphismus]
  Ein Morphismus $f \in \Hom{\C}{A}{B}$ ist ein Isomorphismus, falls ein Morphismus $f^{-1} \in \Hom{\C}{B}{A}$ existiert, sodass $f \circ f^{-1} = \id{B}$ und $f^{-1} \circ f = \id{A}$ gilt.
\end{definition}

Der folgende Beweis für die Eindeutigkeit bis auf eindeutigen Morphismus für initale Objekte kann als Blaupause für strukturell fast identische Beweise für terminale Objekte, Produkte, Koprodukte, uvm.\ verwendet werden.

\begin{theorem}
  Wenn eine Kategorie \C\ ein initiales Objekt enthält, dann ist das initiale Objekt bis auf eindeutigen Isomorphismus eindeutig bestimmt.
\end{theorem}
\begin{proof}
  Seien $A$ und $B$ beides initiale Objekte in der Kategorie \C. Wir müssen zeigen dass \emph{exakt ein} Isomorphismus zwischen $A$ und $B$ existiert.
  \begin{itemize}
  \item Aus der Annahme dass $A$ ein initiales Objekt ist folgt, dass (exakt) ein Morphismus $f$ von $A$ nach $B$ existiert.
  \item Aus der Annahme dass $B$ ein initiales Objekt ist folgt, dass (exakt) ein Morphismus $g$ von $B$ nach $A$ existiert.
  \item Wir müssen zeigen dass $g \circ f \in \Hom{\C}{A}{A}$ der gleiche Morphismus wie \id{A}\ ist. Aus der Annahme dass $A$ ein initiales Objekt ist folgt aber schon, dass exakt ein Morphismus von $A$ nach $A$ existiert, \id{A}, also gilt $g \circ f = \id{A}$.
  \item Analog für $f \circ g$.
  \end{itemize}
  Damit sind $f$ und $g$ die eindeutigen Isomorphismen zwischen $A$ und $B$.
\end{proof}

\section{Produkte und Koprodukte}
\section{Funktoren}

\end{document}

